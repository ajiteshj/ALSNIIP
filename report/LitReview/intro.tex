\section{Literature Review}
\subsection{Analysis of Complex Weighted Social Networks}

In this section the role of edge-weights in a social network is briefly surveyed. The drawback of using plain networks (without weights) is, it does not provide meaning in the real world sense. Therefore the analysis of these types of networks becomes very difficult. Hence it is necessary to assign weights to the ties between the nodes in a network. The weights are assigned to the edges by considering the features of the network and also identifying alternative definitions for the properties of graph theory. The benefits of this improvisation are:-
\begin{itemize}
\item Helps to find better methods for analysis
\item Obtain accurate results after analysis
\item Helps in better visualisation
\end{itemize}

The outcome of this process helps in predicting links and finding influential nodes in an efficient way.

Numerous networked structures can be found in different contexts such as technology, transportation infrastructure, social phnomenon and biological systems. These networks are very complex in nature and are heterogeneous in their capacity and intensity among connections. Before, These features were not considered for studies because links were represented either as present or absent. The two main example networks suitable for study are Scientific Collaboration network(SCN) and World-Wide Air transportation Network(WAN). These networks can be better analysed by assigning weights to the edges proportional to the intensity and capacity of the connections among the nodes. Appropriate metrics are defined to characterize the complex statistical properties and topological observations. The results provides a better insight into the structural heirarchies and descriptions.
The weights are assigned to the edges by considering both the properties of graph theory and the features of the network. The properties are centrality, betweenness, clustering coefficient and power-law behaviour. In some cases it is useful to come up with alternative definitions of centrality, cohesiveness and affinity. The conclusions show a more complete view of the complex networks. The importance of correlation between weights and topology of the networks is appreciated because it provides a better perpective of the network and these details cannot be obtained by the quantities just based on topological information. The study thus offers a quantitative and general approach to understand the complex architecture of real weighted networks.\cite{Barrat2004}

Feature weighting is an important application in content-based recommender systems. In content-based recommender systems, the main attributes are assigned weights. These weights are assigned based on their importance to users. A set of linear regression equations are used to estimate the weight values. These equations are obtained from social network graphs and are judged based on similarity of items.
In content based recommendation system, every item is represented as an attribute or a feature. These features hold numerical or nominal values. The similarity of two items is computed using various distance measures between features. The similarity values are then used to obtain a ranked list of recommended items. The edges are assigned different weights based on human judgement. A formula for computing similarity can be derived by using weights and function of attributes. The feature weights are estimated from social network graphs. A network is constructed using the results of existing recommender with items as nodes. The similarity among items can be induced using optimal feature weights. These optimal feature weights can be determined from linear regression equations. Thus feature weighting provides effective results from the recommender systems leading to rigorous analysis.\cite{Debnath2008}

{\bf Role of edge weights} 

The structure of social networks influence the various dynamic processes of human interactions and communication. Weighted network models help to emulate the structure of real social networks They also help in displaying community structures with weak and strong internal links connecting the communities. The edge weights not only are important for dynamic processes but also in the formation of network structure itself. Weighted social networks can be designed to yield proper weight-topology correlations. They help to generate opinion formation models. Link weights not only make the model more realistic for describing human interactions in a social network but also generate a clear community structure. Thus it can be concluded that interaction weights play an important role in social dynamics. \cite{Toivonen2007}

{\bf Method : Attribute weight assignment}

This component mainly aims to assign a weight to each attribute in a network. It allows to represent attribute importance within a defined context. The assignment of weights depends on the framework created for the network structure. It can be assigned manually or computed automatically. Manual assignment allows users to include their preferences and inputs. Automatic assignment is provide to allow considering the social network characteristics. However, both can be used. The weight assignment to attributes is based on Inverse Functional Property(IFP). An algorithm can be designed for the process of assigning weights to attributes. The important steps of the algorithm are:

\begin{itemize}
\item Computing the importance of each attribute by crawling the related social networks.
\item Convert the input data into useful representation.
\item Computing the similarity score between each pair of similar attributes.
\item Data analysis step
\item Associating each attribute with a set of similarity scores inorder to compute the final weight.
\end{itemize}

Data aggregation/fusion techniques are needed to combine information from different sources and obtain one result for a more accurate decision. These techniques several approaches such as probabilistic models, evidence theories and classical functions. An important application of attribute weight assignment is profile matching in social networks.  It also helps in better understanding of inter-social network operations and functionalities. \cite{Raad2010}

{\bf Example:}
A scientific collaboration network is considered for analysis. The strength of collaborative ties are estimated. A formula for assigning weights to the edges is devised by making suitable assumptions. Social network analysis algorithms are applied on the weighted network and the results are examined. The improvements in the performance of SNA algorithms on weighted networks are observed and reasons for the same are stated. \cite{Newman2001}

@book{ross2014introduction,
  title={Introduction to probability models},
  author={Ross, Sheldon M},
  year={2014},
  publisher={Academic press}
}
