Given a set of weighted, directed relations among the individuals in a large network, clustering based on the fast greedy algorithm  is initially performed on this network to identify closely related sub-networks of individuals. Next, these clusters are analysed one by one as independent organizations. The PageRank  of all the individuals in the cluster are computed in order to determine their local standing (relative to the cluster). For each individual A, the weights that A has assigned to every other individual in the cluster is multiplied by the PageRank of A in order to incorporate the influence of A. With these updated weights, the in-degree centrality (also known as in-ties) is computed and then used as a measure of the total communication directed at each individual. For each individual, the in-tie measure indicates how well other individuals in the cluster weigh this individual. Individuals who receive higher scores are considered more influential in the network.

In order to visualize the distribution of the in-degree measure within an organization, the in-degree scores for all individuals can be sorted in descending order and then graphed, resulting in a ”scree” plot. This technique allows a researcher to get a sense of the distribution of in-ties for all the actors in the organization. Now the focus is on establishing a defensible threshold for identifying the most influential individuals in an organization. Visual inspection of the scree plot can make identifying influentials an onerous task. As an alternative, four reproducible methods are investigated for categorizing influential individuals in an organization. A detailed explanation of these methods follows. 

{\em Method 1 - Absolute Cut Score}

The simplest and most intuitive method for determining a cut score is to set a predetermined absolute value above which individuals are deemed influential and below which they are not. Graphically, this can be accomplished by super imposing a horizontal line over the in-degree scree plot. Those individuals whose influence scores are above the horizontal line are then categorized as influential. However, since this method is based entirely on a single point and is determined independent of variation in the distribution of in-ties, it can result in the situation where every individual (or no individual) in a network can potentially be deemed influential since the criterion is absolute, not relative.

{\em Method 2 - Fixed Percentage of Population}

An alternative method of identifying influentials is to select a fixed percentage of the population as influential. If the top 20\% of individuals in an organization are to be categorized as influentials, this is equivalent to selecting the left-most 20\% of the individuals in the graph. Those individuals to the left of a vertical line superimposed over the scree plot are categorised as influential. As with the Absolute Cut Score, this method identifies individuals as influential independent of the variation of in-ties. It ensures that a given percentage of individuals in the organization are identified as influential and their identification is based upon their performance relative to the performance of other individuals in the organization.

{\em Method 3 - Standard Deviation}

Unlike the first two approaches, the Standard Deviation method focuses on the variation in the distribution of ties. This procedure requires calculation of the mean and standard deviation of the number of in-ties. Then we create a horizontal line two standard deviations above the mean, which can be superimposed over the scree plot. Even though this horizontal line approach is similar to the Absolute Cut Score (Method 1), it however,  does not choose a cut score a priori, instead it utilizes the observed data in determining where to set the cut point. Under this method, those individuals whose in-degree scores are above this line are marked ”influential.”

{\em Method 4 - Random Permutation}

Through the use of random permutations, Method 4 produces results which identify those individuals who received significantly more in-ties than would have occurred by chance alone. This method capitalizes on the creation of a sampling distribution of potential networks that could have occurred, conditional on the fixed row marginals or (out-tie distribution). In order to obtain a sampling distribution of influence for the network, the graph (nodes and edges) is modelled as an exponential random graph. Then the edges are randomly reassigned to individuals in the network keeping the out-degree distribution fixed. By performing one thousand such random permutations the sampling distribution of influence (that is, in-degree distribution) is derived under conditional independence. The ties are not completely independent, as we restrict their new random locations to only emanate from their original sources in the actual data (i.e. the row marginals are fixed). However, in forcing this restriction, we are able to create a sampling distribution of influence that is comparable to our actual data. The result is the distribution that would arise by random chance, given the set of survey responses, and therefore can be used to identify those individuals whose influence is statistically greater than random chance. Once this is completed, individual influence scores are recalculated according to the in-degree measure described earlier. If an individuals actual influence score is higher than his/her ranked counterpart for 95\% of the random iterations, then the individual is labelled a significant influential at the $p > .05$ level.

\subsection{Time Series Analysis}

Time series analysis is an approach where data or datasets are analysed over discrete intervals of time such that processed data for a period of time say t0 serves as an input to time slot t1. The efforts here are to use novel methods so as to analyse the network, establish relations among nodes, predict and forecast the behaviour of social networks over a given period of time. With the help of temporal analysis methods and the underlying algorithms supporting them temporal distances and metrics are determined. 

Time series analysis consists of various steps namely estimation, prediction, query processing, result analysis and stimulation in order to get precise and aesthetically better results. It is used in engineering, business activities and sees application in various fields. Here, the attempt is to try and explore how a given social network reacts over time and how it behaves with the addition of new users, or which users tend to interact among each other, estimation of likelihood that a person X may collaborate with another person Y in future.

%%% ----------------------------------------------------------------------


%%% Local Variables: 
%%% mode: latex
%%% TeX-master: "../thesis"
%%% End: 
